\documentclass[]{article}

%opening
\Huge\title{STAT 4352 - Mathematical Statistics Notes}
\Large\author{JaimeGoB}
\usepackage[margin=0.5in]{geometry}
\usepackage{physics}
\usepackage{amsmath}
\usepackage{centernot}
\usepackage{soul}
\setul{}{1pt}

\begin{document}

\maketitle

\newpage
\Huge\section{Chapter 11 - Interval Estimation}
\Large\textbf{Point Estimators}
\newline $\theta$ is a unknown parameter (feature of a population)
\begin{itemize}
	\item Ex: population mean $\mu$
	\item \textbf{Fixed.} \newline
\end{itemize}
$\hat\theta$ is a point estimator of $\theta$ (it is a numerical value) 
\begin{itemize}
	\item Ex: sample mean $\bar{x}$
	\item \textbf{Varies from sample to sample.}
	\item No guarantee of \textul{accuracy}
	\item Must be \textit{supplemented by} Var($\theta$)
	\newline\Large\rule{1.3cm}{0pt} Standard Error SE($\hat\theta$) measures how much $\hat\theta$ varies from sample to sample.
	\newline\Large\rule{1.3cm}{0pt} small SE $\implies$ low variance thus a more reliable estimate of $\theta$ \newline
\end{itemize}
\Large\textbf{Interval Estimators} 
\newline
\newline\Large\textbf{Interval Estimate}
\newline Provides a range of values that best describe the population.
\newline Let L = L(x) be the Lower Limit
\newline\Large\rule{0.68cm}{0pt} U = U(x) be the Upper Limit
\newline Both L,U are Random Variables because they are functions of sample data.

%\Large\rule{2.3cm}{0pt} \textbf{fixed}
 %\textbf{(Unbiased Estimator)}\newline
 
\section{}

\end{document}




