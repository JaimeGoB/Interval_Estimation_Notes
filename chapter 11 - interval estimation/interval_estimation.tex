\documentclass[]{article}

%opening
\Huge\title{STAT 4352 - Mathematical Statistics Notes}
\Large\author{JaimeGoB}
\usepackage[margin=0.5in]{geometry}
\usepackage{physics}
\usepackage{amsmath}
\usepackage{centernot}
\usepackage{soul}
\setul{}{1pt}

\begin{document}

\maketitle

\newpage
\Huge\section{Chapter 11 - Interval Estimation}
\Large\textbf{Point Estimators}
\newline $\theta$ is a unknown parameter (feature of a population)
\begin{itemize}
	\item Ex: population mean $\mu$
	\item \textbf{Fixed.} \newline
\end{itemize}
$\hat\theta$ is a point estimator of $\theta$ (it is a numerical value) 
\begin{itemize}
	\item Ex: sample mean $\bar{x}$
	\item \textbf{Varies from sample to sample.}
	\item No guarantee of \textul{accuracy}
	\item Must be \textit{supplemented by} Var($\theta$)
	\newline\Large\rule{1.3cm}{0pt} Standard Error SE($\hat\theta$) measures how much $\hat\theta$ varies from sample to sample.
	\newline\Large\rule{1.3cm}{0pt} small SE $\implies$ low variance thus a more reliable estimate of $\theta$ \newline
\end{itemize}
\Large\textbf{Interval Estimators} 
\newline
\newline\Large\textbf{Def: Interval Estimate}
\newline Provides a range of values that best describe the population.
\newline Let L = L(x) be the Lower Limit
\newline\Large\rule{0.68cm}{0pt} U = U(x) be the Upper Limit
\newline Both L,U are Random Variables because they are functions of sample data.
\newline
\newline\Large\textbf{Def: Confidence Level / Confidence Coefficient}
\newline Is the probability that the \textbf{interval estimate} will include population parameter $\theta$.
\begin{itemize}
	\item Sample means will follow the \textul{normal probability distribution} for large sample sizes (n $\ge$ 30)
	\item For small sample  forces us to use the \textul{t-distribution} probability distribution(n $<$ 30)
	\item \textul{A confidence level of 95$\%$} implies that \textbf{95$\%$ of all samples would give an interval that includes $\theta$, and only 5$\%$of all samples would yield an erroneous interval.}
	\item The most frequently used confidence levels are 90$\%$, 95$\%$, and 99$\%$ with corresponding Z-scores 1.645, 1.96, 2.576.
	\item The higher the confidence level, the more strongly we believe that the value of the parameter lies within the interval.
\end{itemize}
\Large\textbf{Def: Confidence Interval}
\newline Gives plausible values for the parameter $\theta$ being estimated where degree of plausibility specified by a confidence level.
\newline
\newline To construct an interval estimator of unknown parameter $\theta$. We must find two statistics \textbf{L} and \textbf{U} such that:
 \[  P \{\textbf{L} \le \theta \le \textbf{U}  \}  = 1 - \alpha  \] 
\begin{itemize}
	\item $P \{\textbf{L} \le \theta \le \textbf{U} \}$ \textbf{Coverage Probability}, in repeated sampling, what percent of samples or Confident Intervals capture true $\theta$.
	\item 100(1- $\alpha$) \textbf{Confidence Interval }- for unknown fixed parameter $\theta$.
	\item L,U - \textbf{Lower and Upper Bounds} - RVs because they are functions of sample data. Vary from sample to sample.
	\item 1-$\alpha$ \textbf{Confidence Level} Probability that estimate will include population parameter $\theta$.
	\item $\alpha$ \textbf{Level of Significance} Percent chance Confidence Interval will not contain population parameter $\theta$.\newline
\end{itemize}
\Large\textbf{Def: Coverage Probability}
\newline $P \{\textbf{L} \le \theta \le \textbf{U} \}$ Gives what $\%$ of samples or Confidence Intervals capture true $\theta$.
\newline
\newline Ex: Coverage Probability = 95$\%$
\newline\Large\rule{1.3cm}{0pt} Will capture $\theta$, 95$\%$ of the time.
\newline\Large\rule{1.3cm}{0pt} Will NOT capture $\theta$, 5$\%$ of the time.
\newline
\newline\Large\textbf{Properties of Confidence Intervals}
\begin{itemize}
	\item Confidence Intervals are not unique.
	\item Desirable to have E[Length of CI] to be small.
	\item If L = -$\infty$ or  U = $\infty$ then we have a one sided interval.
	\item If L,U are both finite, then we have a two sided interval.\newline
\end{itemize}
\Large\textbf{Correctly Interpreting Confidence Intervals}
\newline \textbf{Not Correct}
\newline There is 90$\%$ probability that the true population mean is within the interval.
\newline \textbf{Correct}
\newline There is a 90$\%$ probability that any given Confidence Interval from a random sample will contain the true population mean.
%\Large\rule{2.3cm}{0pt} \textbf{fixed}
 %\textbf{(Unbiased Estimator)}\newline
 
\section{}

\end{document}




